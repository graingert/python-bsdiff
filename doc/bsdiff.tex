\documentclass{manual}
\usepackage[T1]{fontenc}

\title{cx\_bsdiff}

\author{Anthony Tuininga}
\authoraddress{
        \strong{Computronix}\\
        Email: \email{anthony.tuininga@gmail.com}
}

\date{\today}                   % date of release
\release{1.1}                   % software release
\setreleaseinfo{}		% empty for final release
\setshortversion{1.1}           % major.minor only for software

\begin{document}

\maketitle

\ifhtml
\chapter*{Front Matter\label{front}}
\fi

Copyright \copyright{} 2003-2005 Computronix.
All rights reserved.

See the end of this document for complete license and permissions
information.

\begin{abstract}

\noindent
bsdiff is a very simple Python extension module that allows Python to perform
the same tasks as the bsdiff utility available from
\url{http://www.daemonology.net/bsdiff}.

\end{abstract}

\tableofcontents

\chapter{Module Interface\label{module}}

\begin{funcdesc}{Diff}{\var{origData}, \var{newData}}
  Perform a binary diff between the two pieces of data and return a tuple
  consisting of the control tuples, diff block and extra block required to
  reconstitute the new data.
\end{funcdesc}

\begin{funcdesc}{Patch}{\var{origData}, \var{newDataLength},
      \var{controlTuples}, \var{diffBlock}, \var{extraBlock}}
  Return the new data given the original data, the length of the new data and
  the output from the Diff method.
\end{funcdesc}

\chapter{License}

\centerline{\strong{LICENSE AGREEMENT FOR CX\_BSDIFF \version}}

Copyright \copyright{} 2003-2005, Computronix (Canada) Ltd., Edmonton, Alberta, Canada.
All rights reserved.

See the accompanying file LICENSE.txt for details on the license.

Computronix (R) is a registered trademark of Computronix (Canada) Ltd.



\end{document}

